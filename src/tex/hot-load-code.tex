\documentclass{beamer}
\usetheme{Berlin}
\usecolortheme[light,accent=blue]{solarized}
\usepackage{fontspec}
\usepackage{xunicode}
\usepackage{xltxtra}
\setmainfont{DejaVuSans}
\usepackage{booktabs}
\usepackage{lmodern}
\usepackage{listings}
\usepackage{color}
\usepackage{tikz}
\usetikzlibrary{trees, shapes.misc, arrows}
\usepackage{pgfkeys}
\usepackage{graphicx}
\usepackage[normalem]{ulem}
\graphicspath{{./images/}}
\setbeamertemplate{headline}{}

\lstset{%
    basicstyle=\footnotesize\ttfamily,
    breakatwhitespace=false,
    breaklines=true,
    captionpos=b,
    frame=signle,
    keepspaces=true,
    columns=flexible,
    language=Java,
    numbers=left,
    numbersep=5pt,
    numberstyle=\tiny\color{solarizedBase00},
    showspaces=false,
    showstringspaces=false,
    stepnumber=1,
    showtabs=false,
    stringstyle=\color{solarizedMagenta},
    keywordstyle=\color{solarizedCyan},
    commentstyle=\color{solarizedGreen},
    tabsize=2
}

\title{Hot Loading Code}
\subtitle{Using Erlang/Elixir to Deploy Code \\{}
without restarting}
\author[Ballou]{Kenny Ballou}
\institute[/dev/null]{%
    \inst{}%
    /dev/null > labs
}

\AtBeginSection[]{%
    \begin{frame}
    \tableofcontents[
        currentsection,
        sectionstyle=show/shaded,
        subsectionstyle=show/show/hide]
    \end{frame}
}

\begin{document}
% TikZ Styles
\tikzstyle{every node}=[%
    fill=solarizedBase02,
    draw=solarizedBase01,
    thick,
    rounded corners,
    anchor=north,
    sibling distance=6cm]
\tikzstyle{edge from parent}=[%
    solarizedBase00,
    -o,
    thick,
    draw]

\begin{frame}[label=titleslide]
\titlepage{}
\end{frame}

\begin{frame}
\tableofcontents[subsectionstyle=hide]
\end{frame}

\begin{frame}
\frametitle{About Me}
\begin{itemize}
\item{Hacker}
\item{Developer (read gardener)}
\item{Student}
\item{\href{https://twitter.com/kennyballou}{@kennyballou}}
\item{\href{https://github.com}{github/kennyballou}}
\item{\url{https://git.devnulllabs.io}}
\item{\url{https://kennyballou.com}}
\end{itemize}
\end{frame}

\section{Motivation}

\begin{frame}
\frametitle{Hot Loading Code}
\framesubtitle{Problem}
\begin{itemize}
\item{Deploying new code requires downtime}
\item{Downtime can be exacerbated by cold boot problem}
\item{Deployments are error prone}
\begin{itemize}
\item{Especially when achieved manually}
\end{itemize}
\item{Minimizing downtime can be expensive}
\item{Automating deployments \textit{can} be a herculean task}
\end{itemize}
\end{frame}

\begin{frame}
\frametitle{Hot Loading Code}
Using Erlang/OTP, we can:
\begin{itemize}
\item{Load new versions of modules/processes}
\item{Load new versions of application trees}
\end{itemize}
\end{frame}

\begin{frame}
\frametitle{Hot Loading Code}
\framesubtitle{Problems with Hot Loading Code}
\begin{itemize}
\item{Hard to do right}
\item{Difficult to test}
\end{itemize}
\end{frame}

\begin{frame}
\frametitle{Deployment Elsewhere}
\framesubtitle{Related Technologies and Approaches}
\begin{itemize}
\item{Kubernetes\cite{website:kubernetes}}
\item{``Rolling'' Releases}
\item{Blue/Green\cite{website:bluegreendeployment_fowler}}
\item{AWS Elastic Container Service\cite{website:ecs}}
\end{itemize}
\end{frame}

\section{Hot Loading in IEx}

\begin{frame}
\frametitle{Manually Loading New Code}
\framesubtitle{Black Magic}
\begin{itemize}
\item{\texttt{:sys.suspend/1}}
\item{\texttt{:code.load\_file/1}}
\item{\texttt{:sys.change\_code/4}}
\item{\texttt{:sys.resume/1}}
\end{itemize}
\end{frame}

\begin{frame}
\frametitle{Hot Loading Code}
\framesubtitle{Example Code}
\lstinputlisting[firstline=14,lastline=30]{../demo/1/kv.ex}
\end{frame}

\begin{frame}
  \frametitle{Hot Loading Code}
  \framesubtitle{Erlang Functions}
  \lstinputlisting{../demo/1/io/kv.1.out}
\end{frame}

\begin{frame}
  \frametitle{Hot Loading Code}
  \framesubtitle{Example Patchset}
  \lstinputlisting[%
  firstline=15,lastline=25]{../demo/1/patches/01-logging-kv.patch}
\end{frame}

\begin{frame}
  \frametitle{Hot Loading Code}
  \framesubtitle{Erlang Functions}
  \lstinputlisting{../demo/1/io/kv.2.out}
\end{frame}

\begin{frame}
  \frametitle{Hot Loading Code}
  \framesubtitle{Erlang Functions}
  \lstinputlisting{../demo/1/io/kv.3.out}
\end{frame}

\begin{frame}
  \frametitle{Hot Loading Code}
  \framesubtitle{Erlang Functions}
  \lstinputlisting{../demo/1/io/kv.4.out}
\end{frame}

\begin{frame}
\frametitle{Hot Loading Code}
\framesubtitle{Using IEx Helpers}
\begin{itemize}
\item{\texttt{c/1} --- compile}
\item{\texttt{l/1} --- load}
\item{\texttt{r/1} --- (recompile and) reload}
\end{itemize}
\end{frame}

\begin{frame}
\frametitle{Hot Loading Code}
\framesubtitle{Using IEx Helpers}
\lstinputlisting{../demo/1/io/iex.kv.1.out}
\end{frame}

\section{Relups}

\begin{frame}
\frametitle{Hot Loading Code}
\begin{itemize}
\item{Manually swapping code isn't better}
\item{More error prone than cold releases}
\end{itemize}
\end{frame}

\begin{frame}
\frametitle{Hot Loading Code}
\framesubtitle{Release Upgrades --- Relups}
\begin{itemize}
\item{Describe \textit{how} to upgrade Erlang applications}
\item{Uses Erlang release (\texttt{.rel}) files}
\item{Accomplishes upgrading an entire system}
\begin{itemize}
\item{Application upgrades are composed to achieve entire system replacement}
\end{itemize}
\end{itemize}
\end{frame}

\begin{frame}
\frametitle{Hot Loading Code}
\framesubtitle{Elixir Projects}
\begin{itemize}
\item{\texttt{exrm}~\cite{website:bitwalker/exrm}}
\begin{itemize}
\item{Stable}
\item{Requires \texttt{relx}~\cite{website:erlware/relx}}
\item{Poor Umbrella Support}
\item{Actively being replaced}
\end{itemize}
\item{\texttt{distillery}~\cite{website:bitwalker/distillery}}
\begin{itemize}
\item{\sout{Beta} Stable}
\item{Pure Elixir}
\item{Better Umbrella Support}
\item{Bugs}
\begin{itemize}
\item{Supervisors are (still) painful}
\item{May require manual \texttt{.appup} writing}
\end{itemize}
\item{Future}
\end{itemize}
\end{itemize}
\end{frame}

\begin{frame}
\frametitle{Hot Loading Code}
\Huge{Demo}
\end{frame}

\section*{References}
\begin{frame}[allowframebreaks]
\frametitle{References}
\nocite{*}
\renewcommand{\refname}{}
\bibliographystyle{plain}
\bibliography{references}
\end{frame}

\againframe{titleslide}
\end{document}
